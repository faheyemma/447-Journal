% Options for packages loaded elsewhere
\PassOptionsToPackage{unicode}{hyperref}
\PassOptionsToPackage{hyphens}{url}
%
\documentclass[
]{article}
\usepackage{amsmath,amssymb}
\usepackage{lmodern}
\usepackage{ifxetex,ifluatex}
\ifnum 0\ifxetex 1\fi\ifluatex 1\fi=0 % if pdftex
  \usepackage[T1]{fontenc}
  \usepackage[utf8]{inputenc}
  \usepackage{textcomp} % provide euro and other symbols
\else % if luatex or xetex
  \usepackage{unicode-math}
  \defaultfontfeatures{Scale=MatchLowercase}
  \defaultfontfeatures[\rmfamily]{Ligatures=TeX,Scale=1}
\fi
% Use upquote if available, for straight quotes in verbatim environments
\IfFileExists{upquote.sty}{\usepackage{upquote}}{}
\IfFileExists{microtype.sty}{% use microtype if available
  \usepackage[]{microtype}
  \UseMicrotypeSet[protrusion]{basicmath} % disable protrusion for tt fonts
}{}
\makeatletter
\@ifundefined{KOMAClassName}{% if non-KOMA class
  \IfFileExists{parskip.sty}{%
    \usepackage{parskip}
  }{% else
    \setlength{\parindent}{0pt}
    \setlength{\parskip}{6pt plus 2pt minus 1pt}}
}{% if KOMA class
  \KOMAoptions{parskip=half}}
\makeatother
\usepackage{xcolor}
\IfFileExists{xurl.sty}{\usepackage{xurl}}{} % add URL line breaks if available
\IfFileExists{bookmark.sty}{\usepackage{bookmark}}{\usepackage{hyperref}}
\hypersetup{
  pdftitle={My Journal},
  pdfauthor={Emma},
  hidelinks,
  pdfcreator={LaTeX via pandoc}}
\urlstyle{same} % disable monospaced font for URLs
\usepackage[margin=1in]{geometry}
\usepackage{graphicx}
\makeatletter
\def\maxwidth{\ifdim\Gin@nat@width>\linewidth\linewidth\else\Gin@nat@width\fi}
\def\maxheight{\ifdim\Gin@nat@height>\textheight\textheight\else\Gin@nat@height\fi}
\makeatother
% Scale images if necessary, so that they will not overflow the page
% margins by default, and it is still possible to overwrite the defaults
% using explicit options in \includegraphics[width, height, ...]{}
\setkeys{Gin}{width=\maxwidth,height=\maxheight,keepaspectratio}
% Set default figure placement to htbp
\makeatletter
\def\fps@figure{htbp}
\makeatother
\usepackage[normalem]{ulem}
% Avoid problems with \sout in headers with hyperref
\pdfstringdefDisableCommands{\renewcommand{\sout}{}}
\setlength{\emergencystretch}{3em} % prevent overfull lines
\providecommand{\tightlist}{%
  \setlength{\itemsep}{0pt}\setlength{\parskip}{0pt}}
\setcounter{secnumdepth}{-\maxdimen} % remove section numbering
\ifluatex
  \usepackage{selnolig}  % disable illegal ligatures
\fi

\title{My Journal}
\author{Emma}
\date{26/02/2021}

\begin{document}
\maketitle

\hypertarget{workbook-details}{%
\section{Workbook Details}\label{workbook-details}}

This is my workbook.

Week One tasks

\hypertarget{small-heading}{%
\subsubsection{Small Heading}\label{small-heading}}

\begin{enumerate}
\def\labelenumi{\arabic{enumi}.}
\setcounter{enumi}{1}
\tightlist
\item
  Text that is \sout{crossed out}\\
\item
  A link to a \href{windy.com}{fun website}\\
\item
  footnote\textsuperscript{7}\\
\item
  inspirational quote:\\
\end{enumerate}

\begin{quote}
\emph{``If opportunity doesn't knock, build a door''} - \textbf{Milton
Berle}~
\end{quote}

\begin{enumerate}
\def\labelenumi{\arabic{enumi}.}
\setcounter{enumi}{5}
\item
  Bibliographic citation to the start of the best book series
  {[}JKRowling1997HarryPotterAndThePhilosophersStone{]}\\
\item
  Finally, see below record and reflections from week one:~
\end{enumerate}

\texttt{Note\ -\ I\ only\ became\ enrolled\ in\ this\ course\ on\ Thursday,\ so\ have\ been\ unable\ to\ offer\ help\ at\ this\ stage\ and\ did\ not\ attend\ the\ lab.}

Help sought: I struggled to work out how to get Github and Rstudio to
communicate with each other initially, so asked Johannes and Torven for
assistance. They walked me through downloading GitKraken which created
the connection. They also helped me to turn Rstudio into a darker mode,
which looks WAY cooler! Other than that, I have dug through all the
course content to work out how to complete the above exercise.\\

Reflections: It took me a while to work out the terminology/jargon and
correct coding to use, but once I worked out that Knit gave you an
output that showed you where the code wasn't quite working as intended,
it all became MUCH easier! Overall, I can see a lot of different uses
for the programs. I like the idea of being able to see clear version
control when collaborating, seems much clearer and less glitchy that a
google doc!\\
I really hope you're both able to see this once committed to Github, as
I feel I missed a valuable intro in the lecture and workshop! :D

\end{document}
